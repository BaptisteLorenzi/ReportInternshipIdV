\documentclass[12pt]{article}\usepackage{scicite}

\usepackage{times}

% The preamble here sets up a lot of new/revised commands and
% environments.  It's annoying, but please do *not* try to strip these
% out into a separate .sty file (which could lead to the loss of some
% information when we convert the file to other formats).  Instead, keep
% them in the preamble of your main LaTeX source file.


% The following parameters seem to provide a reasonable page setup.

\topmargin 0.0cm
\oddsidemargin 0.2cm
\textwidth 16cm 
\textheight 21cm
\footskip 1.0cm

\usepackage[utf8]{inputenc}
\usepackage{amsmath}
\usepackage{amsfonts}
\usepackage{amssymb}
\usepackage[T1]{fontenc}
\usepackage[left=2cm,right=2cm,top=3cm,bottom=2cm]{geometry}
\usepackage[frenchb]{babel}
\usepackage{graphicx}
\usepackage{caption}
\captionsetup{font=small,skip=10pt}
\usepackage{natbib}
\bibliographystyle{plainnat}
\usepackage{caption}
\usepackage{verbatim,synttree,multicol,booktabs,gb4e}

\setlength{\parindent}{2.5em}
% \usepackage[margin=22mm]{geometry}
\usepackage{fontspec,xltxtra,polyglossia,titling,graphicx}
\usepackage{verbatim,synttree,multicol,booktabs,gb4e}
\noautomath
\usepackage[colorlinks,urlcolor=blue,citecolor=blue,linkcolor=blue]{hyperref}

% The following parameters seem to provide a reasonable page setup.

\topmargin0.0cm
\oddsidemargin0.2cm
\textwidth16cm
\textheight21cm
\footskip1.0cm

%The next command sets up an environment for the abstract of your paper.

\newenvironment{sciabstract}{%
    \begin{quote} \bf}
        {\end{quote}}

% Include your paper's title here

\title{Rapid dynamics in the retina}

% Place the author information here.  Please hand-code the contact
% information and notecalls; do *not* use \footnote commands.  Let the
% author contact information appear immediately below the author names
% as shown.  We would also prefer that you don't change the type-size
% settings shown here.

\author{Baptiste Lorenzi\\
    \\
    \normalsize{CentraleSupélec, Unversité Paris Saclay, Institut de la
        Vision}\\}

% Include the date command, but leave its argument blank.

\date{}

%%%%%%%%%%%%%%%%% END OF PREAMBLE %%%%%%%%%%%%%%%%

\begin{document}

% Double-space the manuscript.

\baselineskip24pt

% Make the title.

\maketitle

\begin{abstract}

sample

\end{abstract}
\textit{During my internship in Olivier Marre's team at l'Institut de la
    Vision, I am
    focusing on computational modeling of the retina. Olivier Marre's team is
    an
    interdisciplinary laboratory, hosting four professors and a dozen 
    interns,
    Ph.D. students and post-docs, working hand in hand to advance research
    on the retina. They all have various backgrounds mainly in biology,
    theoretical physics and engineering. In the context of this project, I've
    been
    working
    closely with Samuele Virgili, a third-year Ph.D. student, whose previous
    and
    current projects all focus on the modeling of retinal ganglion cells.}

\textit{I would like to thank Olivier and Samuele for their kind supervision,
    Déborah for the experimental data,
    as well as the whole team for their
    warm welcome.}
\newline

\tableofcontents

\section{Introduction}\label{sec:introduction}

During my internship in Olivier Marre's team at l'Institut de la Vision, I am
focusing on computational modeling of the retina. Olivier Marre's team is an
interdisciplinary laboratory, hosting four professors and a dozen of interns,
Ph.D. students and post-docs, working hand in hand to advance research
on the retina. They all have various backgrounds mainly from biology,
theoretical physics and engineering. In the context of this project, I've been
working
closely with Samuele Virgili, a third-year Ph.D. student, whose previous and
current projects all focus on the modeling of retinal ganglion cells.

% Context
The ability of the visual system to process complex stimuli on different
temporal and spatial scales is remarkable. % Of which specie? 
Natural environments are such complex stimuli, and extracting the relevant
features at all times is crucial for many species.
% Here I already insist on two notions: natural stim and temporal complexity

% Different neurons at different stages of perception are sensitive to specific
% features of the visual stimulus. From a theoretical point of view, the retina
% doesn't only play the role of a receptor for the visual system. It has been
% proven to be the first layer of feature-sensitive
% neurons \cite{gollisch_eye_2010}.

Both the accessibility and apparent complexity	of the retina make it a
perfect candidate for the study of the front-end of visual processing
\citep{gollisch_eye_2010}. In the mouse, the retina is composed of more
than 30 parallel feature channels, embodied by ganglion cell types. Through
their axons, the optic nerve, they provide information to numerous visual areas
in the brain.
A few channels are active in the encoding of basic features including luminance
changes and motion, that are only combined in more downstream areas. Other
channels however are known to play a role in the extraction of specific
features of natural scenes.
% This example focusing on behaviour is not the best one here...

Still, we currently lack an explanation of the features extracted by other
channels. One of the historical reasons for this is that synthetic stimuli used
to study retinal responses are not complex enough to activate these channels.
Hence, they cannot uncover critical response properties encountered in natural
environments. % Please rewrite this sentence

% Here add an example of how synthetic stimuli have been used to study adaptation

In practice, Karamanlis and colleagues \citep{kim_nonlinear_2020} have
probed a larger complexity in retinal spatial non-linearities thanks to stimuli
capturing the statistics of natural environments.
As those non-linearities cannot be captured by Linear-Nonlinear (LN) models,
convolutional neural networks (CNNs) have become the state-of-the-art approach
for predictive modelling of visual processing, not only in the retina but also
in higher visual areas. % Now briefly explain one or two studies that used CNNs to study adaptation

\textbf{Insight on methods}
Here, we combined the power of CNN-based modeling with large-scale
mutli-electrode recordings from RGCs to investigate the mechanisms of fast
adaptation in the retina under natural stimulus conditions. To this end, we
recorded RGC responses to flashed images paired together. Each pair is composed
of a synthetic adaptation image followed by a natural image. We were able to
identify different trends in the responses of RGCs to natural images, depending
on the adaptation image.

% Need to introduce LSTA

To investigate the diversity of this adaptation process and its implementation,
we paired deep convolutional models with more traditional modeling. We trained
a CNN model on RGC responses to a movie of flashed images. After training, we
study how this model generalized to images after being adapted with
patterns it wasn't trained to. By tweaking part of the model at the inference
level, We hope to show that temporal mechanisms such as gain-control play a
major role in the fast adaptation of RGCs in a natural context.
\section{Background}\label{sec:background}

The retina is part of the central nervous system in vertebrates. It is
made of only a handful of layers of neurons. Its first layer is composed of
photo-sensitive neurons called photoreceptors, that act as light sensors for
the network. They give their excitatory output to bipolar cells, which can be
divided into 14 different types and each type responds differently to the same
stimulus, allowing for a vast functional diversity. Bipolar cells excite in
turn ganglion cells, which finally send the pre-processed visual information to
the rest of the brain through the optic nerve. Ganglion cells can also be
divided into different functional types (at least 32) and each type is believed
to extract a different feature from the visual scene. The retina also has two
classes of inhibitory neurons, horizontal and amacrine cells, that further
modulate the processing of excitatory cells. Compared to the rest of the brain,
its relative simplicity and its relatively easy experimental accessibility make
the retina an ideal neural tissue to study using computational models.

\textbf{Adaptation in the retina}
To operate optimally in a wide range of
stimulation conditions, the retina adapts its responses to the statistics of
the visual scene.
In particular, it was observed to adapt both to the average
luminance (stimulus average) and the average contrast (stimulus distance
from the average or variance).

Visual system can function over a wide range of light intensities, from
starlight to a bright sunny day – a luminance range of 10 10
The retinal adaptation to the luminance of the scene is quite simple by nature.
For instance, it is known that the retina uses different neuronal
pathways at low and high luminance. Rods and their retinal neuronal channels
cover the dimmest light while cones facilitate contrast, color and motion
discrimination but only in brighter light.

Contrast adaptation, by comparison, is harder to study. It was always studied
through the use of simple stimuli.
%clear, need proof
Contrast adaptation is known to have different timescales.
While slower contrast adaptation ($\approx10s$) is better understood, fast
adaptation (<1s) is more complex to study. It is still unclear how it
affects temporal processing and the sensitivity to stimulus
features\citep{baccus_fast_2002}. Furthermore, contrast adaptation can also
happen at different scales, either at the whole scene scale (global contrast
adaptation) or within one ganglion cell receptive field, the part of the visual
field that the cell receives inputs from (local contrast adaptation)
\citep{garvert_local_2013}. Local contrast adaptation is especially relevant in
understanding how ganglion cells respond to natural images since these stimuli
are full of spatial details like edges in which two contrast levels appear
simultaneously. Such images are challenging to use, as they can't be summed up
to a few statistics easily.
\section{Methods}\label{sec:methods}

This work is in many ways a continuation of
\cite{goldin_context-dependent_2022}, from which we derived most of the methods
presented here. By comparison, this time we take into account temporal dynamics
in the responses. The switch from 2D visual inputs to 3D spatiotemporal inputs
complexifies every step of the analysis but is a very stimulating scientific
challenge.

\textbf{Retinal recordings.}
Since I don't realize any experiments myself, I will only give here as few
details as necessary for the understanding of the rest of the work. More on our
experimental strategy is described in \cite{goldin_context-dependent_2022}.

We record the activity of retinal ganglion
cells using a multi-electrode array. A piece of the retina is mounted onto a
membrane. It is then lowered on a 252-channel multi-electrode array whose
electrodes are space by 30 \textmu m.
The data sampling rate was 20kHz.
% TO COMPLETE

% Nat images
\textbf{Natural images.}
We used a subset of the Open Access van Hateren Natural Images Dataset
(https://github.com/hunse/vanhateren). It consists of monochromatic and
calibrated (perfect mapping from
pixel
value to luminance) images of diverse natural environments. These images need
to
be preprocessed to avoid triggering the adaptation to different ranges of light
intensities in the retina, which would trigger unwanted
dynamics. First, images with numerous saturated pixels were not included in our
subset. Using a custom procedure previously developed in the laboratory, we
then
ensured the images were normalized in the mean luminance and the root mean
square (RMS) contrast.

\textbf{Stimulus design.}
The stimuli used in this project are composed of two images, one synthetic
adaptation image followed by a natural image (Figure \ref{fig:LSTA}.a,b).
Adaptation images are taken from
a pool of three different patterns: a grey screen used as control,
a checkerboard of 24*24 checks and the same checkerboard with inverted colors
(Figure \ref{fig:cell_selection}.LSTA). 3180 images were used for
training the CNN, 10 were used to test the CNN and among them, 3 were used to
record an estimation of the LSTA of each cell.
Adaptation and natural images are always paired together to form a single
stimulus pair, also referenced as a clip. Each frame is 864x864 pixels (3.5
\textmu m) wide and each clip is 2*400ms long. %TO CHECK

\begin{figure}
    \centering
    \includegraphics[width=\textwidth]{pics/LSTAExplainV2.png}
    \caption{\textbf{LSTA.}\textit{Adapted from Goldin et al., 2022.} a.
        Natural images are
        perturbed with checkerboards. Scale bar = 500 μm. b. A random sequence
        of
        perturbed natural
        images is being flashed. A gray frame separates all flashes. To compute
        LSTA, we average all different
        perturbations weighed by the number of spikes they evoked. c An example
        of
        LSTA for a cell of the
        retina of a mouse. On top, the spatial and temporal receptive field of
        the
        cell, as is classically used. On
        the bottom, the LSTA of the cell (right) for different natural images
        (left). A green ellipse fitted on
        the classical spatial field is shown on the LSTA for reference. d Same
        as
        c for an axolotl. }
    \label{fig:LSTA}
\end{figure}
% Give details about how the images are built for the camera.

The training set is composed of 3180 clips, each composed of a grey
adaptation image followed by a natural image. The test set is composed of 30
clips, each repeated 30 times, for a total of 900 clips.
The test clips are composed of 10 different natural
images preceded by each adaptation (3 different clips for each natural image).
The dataset used to record LSTA is composed of 9 different clips repeated 1000
times. Each clip is composed of one of the three selected natural images
preceded by one of the adaptation patterns.

We first used 4 different natural images while computing the LSTA of each cell,
each 12 different clips being repeated 12 times. We found that the estimation
of the LSTA was too unstable with only 750 repetitions. In the following
experiments, we excluded the image that yielded the least amount of stable
estimations of LSTA (20\% average success rate as compared to 42\% average
success rate for the other three images). We then used 3 different natural
images while computing the LSTA of each cell, each 9 different clips being
repeated 1000 times. We found that the estimation of the LSTA was more stable
with 1000 repetitions.

\textbf{Data processing.}
Multi-electrode array experimental data takes the shape of a collection of
temporal electrical signals tiling the recorded area.
In most scenarios, including here, these signals are sorted into different cell
signals using a semi-automatic clustering algorithm. This algorithm is based on
the shape
of the electrical spikes as well as their spatial location. It is
quite messy due to the low signal-to-noise ratio in the data and each
experiment needs to have its sorting corrected by hand.
This process can take up to an
entire day for a single experiment. I used spiking-circus for semi-automatic
spike sorting and the UI phy for handmade corrections \citep{yger_spike_2018}.
It is important to note that even though the retina is an easier organ than
most to record clean spike signals from, the data is still very noisy and the
sorting process is not perfect. Hence, when validating hypotheses, cells are
usually rated by their reliability.

% Checkeckerboards and RF
After spike sorting, we analyze the recording from standard stimuli to
characterize each
ganglion cell receptive field. To this end, we display a random binary
checkerboard for approximately 1 h at 30 Hz. Check size is 42\textmu m. A
ganglion cell receptive is computed as its spike trigger average (STA), for
this checkerboard stimulus. The STA of a cell can also be described as the
stimulus that triggers the most spikes from that cell. It is computed as the
average of the presented checkerboard weighted by the number of spikes using a
set number of samples per repetition (here 21). The spatial STA is usually
shown as the 2-dimensional spatial slice at the maximum value after smoothing.
The temporal STA is the one-dimensional time slice at the pixel with the
maximum
value. For smoothing, a double Gaussian is fitted on the resulting spatial
STA.

% Add a figure with a clear viz of STA and RF
% Cell typing (probably useless here, but say a word about it)

\textbf{LSTA.}
To record the local specificity of the response of a ganglion cell to a natural
image, we used a method called local spike trigger average (LSTA) for its
analogy with the STA. This method was previously developed in the laboratory.
We first generate a set of perturbed natural images by superimposing some of
the natural images (3-4 images) with various perturbation patterns in the form
of random checkerboards (Figure \ref{fig:LSTA}). We once again used a checker
size of 42\textmu m.
Following calibration guidelines measures in previous experiments, the
amplitude of the perturbation was set to 12.5\%, where 100\% corresponds to a
pixel value of 1. In the mouse retina, this amplitude was found to trigger a
change in firing rate of approximately 1.5Hz in ganglion cells with high firing
rates to the unperturbed images \citep{goldin_context-dependent_2022}. To avoid
any adaptation to a single reference natural image, we shuffled randomly all
the clips.

As for the STA, the LSTA is computed as the average of the perturbation
patterns weighted by the response of the RGC. The responses are computed by
counting the number of spikes on a time window ranging between +50ms and +450ms
after the display of the natural image. It was chosen to minimize the impact in
the response of delayed spikes due to the perturbation patterns while
maximizing
the capture of spikes that are responses to the natural image.

% FROM NOW ON THERE IS S LOT TO WRITE...
\textbf{Data visualization.}
Due to the complexity of the recorded information, it can be challenging to
have
all the relevant information on one screen. First, it's important to always
have the raster plot in sight since it informs of the quality of the cell at a
glance (Figure \ref{fig:cell_selection}).
Then, we usually display the STA and the LSTA of the cell, typically zoomed on
the cell receptive field (Figure \ref{fig:CellExample}).
Finally, we display the temporal profile of the cell response as the
Poststimulus time histograms
(PSTH). They are histograms of the times at which neurons fire.

% IS THIS PART VAUABLE......

% Add a figure that sums up all the visualization, like a cell summary but with absolutely everything !!!

\textbf{Cell selection.}
Not all recorded cells are suitable for all types of analysis. Overall, the
main quality we look for in a cell is its stability during an entire stimulus
clip.
It reflects its health and likeliness to behave normally during the
presentation. For our most complex experimentations, cells need to remain
stable in vitro for up to five hours, which is quite unlikely. In Figure
\ref{fig:cell_selection}, we
describe the different selection steps. Among the 264 cells
in our last experiment, only 14 passed all of them. Note that a cell doesn't
have to
pass all the tests to be considered for a given analysis. For example, any cell
that passed the 'CNN-Train' could be used to fine-tune the CNN model.
% Graph idea : All types of rasters for one cell

% I got a bit original on that figure, I don't think it's that great but well
\begin{figure}
    \centering
    \vspace*{-3cm}
    \includegraphics[width=0.8\textwidth]{pics/CellSelectionV2.png}
    \caption{\textbf{Selecting cells suitable for the different degrees
            of
            analysis.} \textbf{Middle.} Flow chart of the cell selection. In
        green, we painted the amount of cells that successively passed each test.
        \textbf{Outer.} Examples of good and
        bad cells
        for each test. On each raster plot, a vertical bar represents the
        instant the natural image was shown, following the adaptation pattern.
        For the
        training dataset, we looked for cells that
        consistently spike on natural images. The more constant the delay, the
        more
        stable the cell. For the test dataset, we looked for cells that gave a
        stable
        response to most of the repeated test clips. For the LSTA analysis, we
        looked
        for cells that showed bright spots in the unfiltered data, to avoid
        confusing
        noise for a signal after smoothing. To determine if the cell had a
        modeling
        interest, we smoothed the LSTA and zoomed on the cell receptive field.
        A cell
        is considered interesting if its LSTA is well-defined for numerous
        pairs. The
        most interesting cells are those where the LSTA is impacted by the
        adaptation
        image.}
    % That description is not clear enough if the rasters are not described somewhere else
    \label{fig:cell_selection}
\end{figure}

% Models: Put the emphasis on the balance between G.C. and data-driven models
% Also insist on the time component of the models

\textbf{Modeling.} This is the most consistent part of my work in the
laboratory, as well as the
most challenging. We are designing a dynamic model of the retinal fast
adaptation.
By comparing how different modeling strategies reproduce the
observed LSTA in the data, we can gain insight into how fast adaptation to
natural scenes is implemented in the retina.
%I think what is missing here is what question you want to answer with this. We should discuss this more. 

We first created a toy model (data-agnostic) of the retina, based on the LNLN
model described
in \ref{sec:background}. By adding a gain control mechanism to the model, we
were able to reproduce some behavior of the LSTA as observed in the data
(Appendix \ref{chap:toyModel}).
This approach is very limited since the numerous non-linearities in the LNLN
the model makes its dynamics very complex.
As an answer, we are fitting a more complex model, based on a convolutional
neural network (CNN), that will be able to infer the non-linearities from the
data. For this report, we'll focus on this part of the modeling since it is
what I've been working on the most.
We are using a deepnet framework, optimized using the
Pytorch library. In practice, we use a model that differs slightly from the
LNLN
model. In Figure \ref{fig:CNN_simple}, we show a simplified view of our
architecture focusing on how the model predicts spikes from a video clip.

Additionally, we are using a few regularization strategies to improve the model
performance as well as ensure it learns filters that are comparable to the
retina. % CHANGE THAT
The subunit filters are regularized using a smoothing constraint, the L2
regularization. The RGC receptive field is regularized using a sparsity
constraint, the L1 regularization, as well as L1. Using a process called batch
normalization, we also ensure that the activation maps as well as the feature
scores matrix stay around 0 while not taking too small values that would
produce numerical errors. Finally, to partially avoid overfitting on the
training set, we added a dropout layer between the subunits and the RGC
receptive field.

Each trial of the model was optimized using the Adam optimizer, with varying
learning rates. The loss function is the Negative log-likelihood loss with
Poisson distribution of target. It is a standard loss function for predicting
rare events such as neural spikes.
The hyperparameters of the regularizations were optimized using a random grid
search strategy, using a pipeline recently developed within the laboratory.
% ADD REFERENCE

All the code was written in Python and is mainly built around the Pytorch
library \citep{pytorch}.
We thought the explanations of those concepts were out of the scope of the main
text of this report. A more accurate and mathematical description of the model,
the regularizations and the optimization strategy can be found in the Appendix
\ref{app:CNN}.

As an indication, one trial of the model takes approximately 2 hours to train.
The current results of this project are already the results of more than one
month of consecutive computation.

\begin{figure}
    \centering
    \includegraphics[width=\textwidth]{pics/CNNSimpleWithImgs.png}
    \caption{\textbf{From stimulus to prediction in the CNN model.} Elements in
        black are the input and consecutive outputs while colored
        elements
        represent the model layers. Lines with arrows represent a convolution
        (the
        filter is dragged upon the input to produce the output, usually of
        smaller
        dimensions), while pointed lines represent a zoom on a given layer. N
        is the size of the batch. The subunit filters and the ganglion feature
        weights layer end
        with a non-linearity (ReLU and SoftPlus respectively).
        Each subunit filter (here only two are shown) represents the
        selectivity
        of a subunit (one can think of them as bipolar cells). Each subunit
        type covers the entire input spatially. They also integrate temporal
        events on a given number of frames. Each pixel of the
        activation maps encodes the level of activation of that subunit in that
        particular location and at that particular instant.
        The RGC receptive field is similar instead that it tiles the entire
        spatial input. It could be described as a pooling of the subunit
        activation
        maps. The RGC is modeled to also integrate temporal events on a given
        number of frames. This spatiotemporal receptive field is the same for
        all
        subunits but the scores corresponding to each subunit pathway are
        different.
        Finally, the feature weights represent how much of each feature is
        relevant foe
        determining the spiking frequency on a given temporal bin. This layer
        is just
        a weighted average of the two features and a non-linearity.
    }
    \label{fig:CNN_simple}
\end{figure}

\textbf{Model evaluation.}
To evaluate the performance of the models, we used a testing set of 30
different stimuli where each stimulus has been repeated 30 times as described
in \textbf{Stimulus design}.
We computed single-trial correlations $\rho$ between the ground truth and the
prediction following the following steps:
\begin{enumerate}
    \item We defined the ground truth as the average firing rate over all
          repetitions of the same stimulus.
    \item We standardized both the ground truth and the prediction by
          subtracting the mean and dividing by the standard deviation of each group
          respectively. This is to take into account that the output of the model is not
          necessarily on the same scale as the ground truth.
    \item We computed the Pearson correlation between the two vectors.
\end{enumerate}
% I can add this if I have time to actually do it before Sunday

\clearpage

\section{Results}
\label{sec:results}
Some idea of speech: *** responses of exemplary RGCs

Here, we investigated fast adaptation in the mouse retina under natural
stimulus conditions. To this end, we trained a CNN model on RGC responses to a
movie of flashed images appearing naturally in the mouse environment, % and then performed a model-guided
% search for stimuli that maximise the responses of RGCs.

\textbf{A method to estimate how selectivity to natural images changes over time.}
We recorded retinal ganglion cells (RGCs) in the mouse retina with multi-electrode arrays (MEAs) while displaying sequences of natural images.
Each image was presented for 400 ms, preceded by one of three 400 ms adaptation light patterns: grey, checkerboard, or inverted checkerboard ADD FIG SUMMARY OR ID CARD.
Each pair of adaptation pattern and natural image forms a stimulus clip lasting 800 ms.
To measure the selectivity of RGC to different parts of the image, we added dim checkerboard patterns (Fig. \ref{fig:LSTA}).
The amplitude of the pertrbation checkerboard was selected to introduce a small yet visible change in the RGC response compared to the RGD response to unperturbed natural image.
1000 repetitions with different perturbation patterns was necessary to estimate the LSTA to one clip (see Methods).

For each cell and each stimulus clip, we computed ane stimation of a local
spike-trigegr average (LSTA) (Figure \ref{LSTA}), as the average of the
perturbation patterns wieghted by the number of splikes they evoked. This
estimation is similar to a more classical Spike Trigger Average (STA), but due
to small amplitude of the perturbation checkerboard, we explore here a small,
local region of the stimulus space centered on the reference natural image. The
LSTA is a visualization of the gradient of the RGC response at the reference
natural image point in stimulus space. From an experimental point of view,
instead of perturbating the biological system itself (e.g. shutting down
neuronal pathways) we perturbated the stimulus itself.

We recorded RGC response from four different eyes. The first experiment was
discarded since 750 repetitions was not sufficient to esimate the LSTA. The
second experiment was also discarded since the retina was very unhealthy during
the recordings. The third experiment was a great success with over 100 cells
showing relevant LSTA for many different clips. FInally, we used a fourth
experiment to both train and test our model and measure LSTA, once again with
great success with about 100 cell with LSTAs despite the longer experimental
time.

\par~\textbf{Ganglion cells can change their selectivity depending on previous
    light patterns.}

We looked at the firing rate and the LSTA of more then 200 cells changed
depending on the adaptation light pattern. Some clear hyptohesis can be made as
in describe in Figure \ref{fig:CellExample}. We observed We have yet to do a
statiscal analysis of the number of occurrences of those different behaviour.

\textbf{Can a convolutional neural network model account for this dependance?}

\textit{Disclaimer: This part is a work in progress.}

\begin{figure}
    \centering
    \includegraphics[textwidth]{pics/ExampleCells.png}
    \caption{\textbf{PSTH and LSTA depend on previous light patterns.} \textbf{Left} Poststimulus time histogram (PSTH) are histograms of the times at which neurons fire. In each plot, the time bins in grey correspond to the display of an adaptation pattern while time bins in white natural images.
        The time axis is centered on the moment when the current natural image is displayed. So in order the four rectangles are previous natural image, current adaptation pattern, current natural image, next adaptation pattern. The area highlighted in yellow corresponds to the 400 ms windows over which spikes are integrated to compute the LSTA. Adaptation effect can already be seen there, the fring rate of the RGC can be modified by the adaptation pattern.
        Each color corresponds to a different 'previous natural image' displayed 800ms before the current natural image. We noticed that it had an influence on the response of the RGC to the current natural image. After looking at all the pSTH of all the cells, we judged that it was minor enough to not impact the behavior of the RGC that we want to visualize. Longer time windows would reduce the number of repetation that can be shown in a set amount of time and reduce the quality of the estimation of the LSTA.
        \textbf{Right} Local Spike Triggered Average (LSTA), average of the perturbation checkerboard patterns weighted by the number of spikes they evoked. Here we centered the view on the RGC receptive field and smoothed the LSTA using exponential tuning and spatial interpolation. We observed dependence on the adaptation pattern in On (Cell 1 ...), Off (Cell 2) and OnOff cells (Cell 3). A first hypothesis would be that for On cells, an area that stays white provoke less response that an area that goes from black in the adaptation pattern to white in the natural image - respectively black for Off cells (On Cell - Column ...).
        However, this hypothesis does not hold true in all cases (jjj). For OnOff cell, this adaptation can happen in both pathways simultaneously (cell OnOff). We observed that a high spiking rate was not necessarily linked with a clearly defined LSTA and vice-versa. }
    \label{fig:CellExample}
\end{figure}

\section{Conclusion}\label{sec:Conclusion}

We combined large-scale recordings of RGC responses to natural movie
stimulations with CNN-based modeling to investigate the mechanisms of fast
contrast adaptation in the retina.

The modeling of retinal responses to natural stimuli has improved our
understanding of complex retinal processing. In a recent review, Karamanlis and
colleagues [ADD CITE], suggested three perspectives of study on the retinal
encoding of natural scenes: The circuit perspective ('How is the retinal code
implemented?'), the normative perspective ('Why is it complimented this way?)
and the coding perspective ('What is the code used by the retina?'). In this
work,
We focus on the 'what'. By exploring the response of the retina to a portion of
the spatio-temporal stimuli space we can gain insight into the code used by the
retina on that subspace. To explore further the 'how' perspective, one would
need to study how the different known types of cells in the retina participate
in that encoding. This poses the challenge of bridging the typing of cells from
functional and anatomical perspectives.
The normative perspective has also been explored using deep CNNs with
anatomically realistic constrained. It is likely that species with simpler
cortical circuitry, as mice, have a stronger need for upstream feature
extraction, in the retina. In opposition, species with computationally powerful
cortexes such as primates can deal with more faithful and linear
representations
of the visual inputs.
Some studies admirably developed approaches that allow investigation of retinal
processing from all three perspectives [ADD CITE].

% Your references go at the end of the main text, and before the
% figures.  For this document we've used BibTeX, the .bib file
% scibib.bib, and the .bst file Science.bst.  The package scicite.sty
% was included to format the reference numbers according to *Science*
% style.

%BibTeX users: After compilation, comment out the following two lines and paste in
% the generated .bbl file. 

\bibliography{biblio}
\newpage
\appendix
\setcounter{equation}{0}      
\section{Convolutional Neural Network model (CNN)}\label{app:CNN}

\textbf{Preprocessing.}
We performed the following preprocessing steps on the dataset:
\begin{enumerate}
    \item Each frame was cropped on half of the image since the rest is
          completely out of the retina field of view (frame dimension: 864 →
          432
          pixels / 3024 → 1512 microns).
    \item It was downsampled by a factor of 4 (frame dimension: 432 → 108 pixels /
          3.5 → 14 microns per pixel)
    \item Each pixel value was standardized by subtracting the mean and
          dividing
          by the standard deviation of the pixel values of the whole dataset.
    \item Spikes during the presentation of the natural image was binned into
          16 bins of 25ms to produce a 16-bin PSTH for each mini-movie. Spikes
          during the
          adaptation images aren’t used at all.
    \item 80\% of the data was used for training, and 20\% for validation.
          Testing was done on a separate dataset.
\end{enumerate}

\textbf{Model architecture.}
The first layer of the CNN model was a convolutional layer with S
two-dimensional convolutional kernels, where the filters were learned from the
data.
The second layer was composed of a single three-dimensional filter and a vector
of feature weights (with one weight
for each of the features extracted by the first layer) followed by a
non-linearity.
Using a common filter for all the features helps reduce the number of
parameters, following previous work \citep{cadena_deep_2019}.

The model is composed of:
\begin{enumerate}
    \item Convolutional kernel weights $k_{rstk}$ that compute convolutions
          with
          the input images (where $r$ and $s$ index the single kernel spatial
          dimensions, $t$ indexes time,
          and $k$ indexes kernels).
    \item Pointwise nonlinear functions
          $f_{\theta_k^{[1]}}$ that convert the convolutional outputs
          into non-negative activation values.
    \item In addition, for each neuron $n$:
          \begin{enumerate}
              \item Readout weights $w_{ijtk}$ which can be factorized as
                    $w_{ijtk} =
                        u_{ijt}v_{kn}$, where $i$ and $j$ index space, $t$
                    indexes time, $u_{ij}$ represent the spatial
                    weights, and $v_{k}$ the feature weights.
              \item A pointwise nonlinear function
                    $f_{\theta^{[2]}}$.
          \end{enumerate}
\end{enumerate}

We choose $f_{\theta_k^{[1]}}$ to be ReLU functions ; \[
    \text{ReLU}(x) =
    \begin{cases}
        x, & \text{if } x \geq 0 \\
        0, & \text{if } x < 0
    \end{cases}
\].

$f_{\theta_n^{[2]}}$ was selected to be a softplus function :
$\text{softplus}(x) = \ln(1 + e^x)$. It was used as a smoothed ReLU.

In practice, we fix the nonlinearities in the implementation and learn their
threshold in the model as bias in the convolutional layer, which is strictly
equivalent to learning the threshold of the nonlinearities.

The outputs (activation map) of the $k$th unit of the first layer
were:
\[
    A_k = f_{\theta_k^{[1]}} (\sum_{k} K_k \cdot X) \
\]

The spiking rate $r$ of the neuron given an input image $X$ on the temporal bin
number $n$
was:
\[
    r_n(X) = f_{\theta_n^{[2]}} (\sum_{k} \sum_{ij,t \in [t_0 - 7, t_0]}
    u_{ijt}v_{kn}A_{ijtk})
\]

Additionally, batch normalization was applied to the outputs of the first
layer as well as between the filter and the feature weights of the second
layer.

Batch normalization normalizes the activations in a batch of data
during training. It is typically applied before the activation function.
The goal is to mitigate issues related to internal covariate
shifts and facilitate faster convergence.

In batch normalization, given a batch of inputs $X = \{x_1, x_2, \ldots,
    x_m\}$ for a layer, the following steps are performed:

1. Compute the mean and variance of the batch:
\[
    \mu = \frac{1}{m} \sum_{i=1}^{m} x_i \quad \text{and} \quad \sigma^2 =
    \frac{1}{m} \sum_{i=1}^{m} (x_i - \mu)^2
\]

2. Normalize the batch using the mean and variance:
\[
    \hat{x}_i = \frac{x_i - \mu}{\sqrt{\sigma^2 + \epsilon}}
\]

Where $\epsilon$ is a small constant added for numerical stability.

3. Scale and shift the normalized values with learnable parameters:
\[
    y_i = \gamma \hat{x}_i + \beta
\]

The parameters $\gamma$ and $\beta$ are learned during training through
backpropagation.

\textbf{ Regularization on Convolutional Kernels}

We employed a L2 regularization applied to the convolutional kernels of the
first layer:

\[
    L_{2^{[1]}} = \lambda_{2^{[1]}} \sum_{i,j,t} (\lvert k_{ijt} \rvert^2)
\]

Here, $K_{rsk}$ represents the convolutional kernels, $\epsilon = 10^{-8}$ is a
small constant, and $p$ sums over all the relevant indices.

The L1 and L2 regularizations applied to the spatial weights and feature
weights of the second layer are given by:

\[
    L_{1^{[2]}} = \lambda_{1^{[2]}} \sum_{i,j,t} \lvert u_{ijt} \rvert
\]

\[
    L_{2^{[2]}} = \lambda_{2^{[2]}} \sum_{i,j,t} (\lvert u_{ijt} \rvert^2)
\]

Here, $\lambda_{2^{[1]}} $, $\lambda_{1^{[2]}}$ and $\lambda_{2^{[2]}}$ are the
regularization hyperparameters,
and $k_{ijt}$, $u_{ijt}$ represent the spatiotemporal weights of the first and
second layer respectively.

\textbf{Model training.}
Considering the m recorded video-response pairs $(X1, y1).., (Xn,yn)$ the
resulting loss function is given by:
\[
    L= \frac{1}{m} \sum_{k=1}^{m} \frac{1}{n} \sum_{l=1}^{m} (r(X_{kl}) -
    y_{kl}
    \log(X_{kl})) + \lambda_{2^{[1]}} L_{2^{[1]}} + \lambda_{1^{[2]}}
    L_{1^{[2]}}
    \lambda_{2^{[2]}} L_{2^{[2]}}
\]

where the first term corresponds to the negative
log-likelihood of the Poisson loss and where $\lambda_{2^{[1]}} $,
$\lambda_{1^{[2]}}$ and $\lambda_{2^{[2]}}$ are the hyperparameters
which controls the importance of the regularization terms. We
fitted the model by minimizing this loss using the Adam optimizer on the
training set.

The batch size was fixed to 64.
The learning rate was gradually reduced using a decay: if for a set number of
epoch the validation score did not improve, then the learning rate was divided
by 0.2.
Each was randomly initialized using He normal initialization :
\[
    \text{He Initialization:} \quad W \sim \mathcal{N}\left(0,
    \frac{2}{n_{\text{in}} + n_{\text{out}}}\right)
\]
$n_{\text{in}}$ is the size of the input (layer 1: number of pixels in the
clip) and $n_{\text{out}}$ is the size of the output (layer 1: number of pixels
in the
activation map).

We cross-validated the hyperparameters $\lambda_{2^{[1]}} $,
$\lambda_{1^{[2]}}$ and $\lambda_{2^{[2]}}$, the learning rate as well as the
size of the filter of the subunits
by performing a random search. To this end, we used a personalized pipeline
currently in development in the laboratory based on the Optuna library \cite{}
[CITE OPTUNA]. The optimal hyperparameter values were the ones
whose model produced the lowest loss value (without regularization terms) on
the validation dataset.

\section{A toy model of gain control}\label{app:toyModel}

\begin{figure}
    \centering
    \begin{subfigure}{.5\textwidth}
        \includegraphics[width=0.95\textwidth]{pics/GCModelDiagram.png}
        \label{fig:LNLN}
    \end{subfigure}%
    \begin{subfigure}{.5\textwidth}
        \includegraphics[width=0.95\textwidth]{pics/exModel.png}
        \label{fig:ToyModel}
    \end{subfigure}
    \caption{\textbf{Left: Quick sketch of a gain control LNLN model.} Each
        bipolar
        cell is
        composed of a linear spatial filter that selectively responds to part
        of the scene,
        a non-linear activation function, and a gain control mechanism that
        scale its output
        depending on past events. They all converge into one bipolar cell
        (forming its receptive field)
        of which output is also modeled using a non-linear function.
        textbf{Right.} Recreating displacement of LSTA with a Gain
        Control toy model.}
\end{figure}

\clearpage

% \appendix
\section{Enjeux technologiques, sociaux et éthiques}\label{app:enjeux}

Le travail que j'ai réalisé pendant ce stage est relativement fondamental.
Toutefois, la compréhension des mécanismes de la vision est un enjeu majeur à
la fois pour la recherche en neurosciences et dans une moindre mesure pour la
recherche en intelligence artificielle.

D'un point de vue médicale, le laboratoire dans lequel j'ai effectué mon stage
développe au moins deux projets visant les maladie de la vision.
Le premier en collaboration avec l'entreprise Essilor, vise à coomprendre
comment la rétine rentre en jeu dans le développement de la myopie. C'est un
enjeu médical
et industriel important car la rétine touche aujourd'hui plus de 20\% de la
population mondiale et ce chiffre est en constante augmentation.
Le second projet vise la resaturation partielle de la vision chez des patients
atteints de maladie dégénérative de la rétine (DMLA).
La plus ancienne des stratégies consiste en la pause d'implants rétiniens qui
viendraient copier l'activité du nerf optique.
Malgrè de nombreuse difficultés techniques, cette solution est en phase
d'essais cliniques dans des start up comme Pixium Vision
(https://www.pixium-vision.com/).
Une solution plus récente consiste à rendre sensible à la lumière (laser de
haute puissance) par optogénétique des
cellules ganglionnaires de la rétine.
Toutefois, ces cellules sont beaucoup moins nombreuses que les photorécepteurs
et la résolution de l'image est donc très faible.
Le projet du laboratoire vise à rendre sensible à la lumière des cellules
bipolaires qui sont plus nombreuses et qui sont connectées aux cellules
ganglionnaires.
Cela permettrait d'augmenter la résolution de l'image et donc de rendre la
vision retrouvée plus précise.
Les cellules bipolaires sont malheureusement beaucoup plus difficiles à cibler
car elles ne sont pas en surface de la rétine.
Une nouvelle technologie en cours de développement au laboratoire permettrait
égallement d'activer ces cellules non pas par lumière laser mais par des
vibrations sonores, beaucoup moins énergétiques.
Mes travaux participent entre autres à la compréhension du code neuronale de la
rétine. Un savoir nécessaire pour définir avec quel signal activer ces
cellules.

D'un point de vue technologique, le système cérébral est égallement un des
modéles les plus parfait de traduction de l'information visuelle.
Cette tâche nécessaire à de nombreuses applications en intelligence
artificielle (robotique, imagerie médicale, conduite autonome...), il est
égallement capable de la réaliser avec une efficacité énergétique et une
robustesse inégalée par la machine. La compréhension des mécanismes de la
vision est égallement un élément clef pour le développement de technologies
explicables et sans biais, un enjeu de poids en imagerie médicale notamment.

\section{Bilan de compétences}\label{app:bilan_competences}

Le présent bilan de compétences a pour objectif de récapituler les compétences
acquises et les expériences professionnelles vécues lors de mon stage.

\subsection{Compétences Techniques Acquises}

\begin{enumerate}
      \item \textbf{Analyse de Données:} J'ai acquis des compétences avancées
            en
            analyse de données de multi-eletrode array (MEA). J'ai appris à
            utiliser des outils tels que
            spiking circus, phy et la pipeline du laboratoire pour le
            prétraitement et l'analyse statistique des
            données. J'ai également contribué directement à l'amélioration de
            cette pipeline.

      \item \textbf{Modélisation Computationnelle:} J'ai développé des
            compétences en modélisation de réseaux neuronaux à l'aide de
            logiciels comme Pytorch.
            J'ai également
            travaillé sur l'implémentation de modèles mathématiques pour
            simuler
            des phénomènes neurologiques ayant un aspect à la fois spatial et
            temporel.
            J'ai également participer à la mise en place d'une pipeline
            d'optimisation d'entrainement de modèles
            (calcul parrallèle sur GPU, sauvegarde des modèles, sauvegarde des
            résultats...).

      \item \textbf{Programmation:} J'ai amélioré mes compétences en
            programmation et en visualisation des donnés, en particulier en
            Python et en utilisant des
            bibliothèques
            telles que NumPy, TensorFlow et PyTorch pour la mise en œuvre de
            modèles et d'algorithmes.

      \item \textbf{Analyse Statistique:} J'ai acquis une solide compréhension
            des méthodes statistiques utilisés pour évaluer les modèles de
            prédiction
            de réseaux de neurones biologiques (modèles de poisson, single-triral
            correlation...).

      \item \textbf{Communication Scientifique:} J'ai perfectionné mes
            compétences en rédaction de rapports scientifiques, en préparation
            de
            présentations, et en communication des résultats de mes recherches
            de
            manière claire et accessible. Durant mon stage j'ai participé au
            congrès junior
            de l'Université Paris Saclay. J'ai également pu présenter des
            travaux
            antérieurs au stage
            à l'Institut du Cerveau ainsi qu`à la Journée du département Science
            des Données du plateau de Saclay.

      \item \textbf{Culture scientifique:} Je suis resté dans une optique
            d'apprentissage constant en lisant
            de nombreux articles scientifiques et en participant à des
            séminaires
            et des
            conférences, notamment une semaine à l'"European Retina Meeting".
            J'ai également participé à des réunions de laboratoire
            hebdomadaires.

      \item \textbf{Suivi de projet de longue durée:} J'ai appris à gérer mon
            temps et à travailler de manière autonome sur un projet de longue
            durée, bien souvent en l'absence de deadline et d'objectifs
            concrets.
            Pour cela j'utilise dfférents outils de gestion de projet comme la
            tenue d'un journal, d'un agenda et d'un wiki personnel.

\end{enumerate}

\subsection{Expériences Clés}

\begin{enumerate}
      \item \textbf{Projet de Recherche:} J'ai participé activement à un
            projet
            de recherche sur la modélisation fonctionelle de la rétine, en
            contribuant à la conception
            expérimentale, à l'analyse des résultats et
            à la modélisation des observations.

      \item \textbf{Collaboration Interdisciplinaire:} J'ai eu l'occasion de
            collaborer avec des chercheurs issus de disciplines diverses comme
            la biologie, l'informatique et la physique, ce qui
            m'a permis
            d'élargir mes horizons et d'apporter une perspective complémentaire
            à mon travail.

      \item \textbf{Travail en Équipe Internationale:} J'ai travaillé en
            collaboration
            avec d'autres membres de l'équipe de recherche, ce qui m'a permis
            d'apprendre à
            résoudre des problèmes de manière collective et de développer des
            compétences interpersonnelles. J'ai appris à utiliser l'anglais au
            travail de manière quotidienne pour m'adapter à l'aspect
            international de l'équipe.
\end{enumerate}

\subsection{Conclusion}

En conclusion, mon stage de recherche de fin d'étude m'a
permis d'acquérir des compétences techniques avancées et une expérience
précieuse dans le domaine des neurosciences computationnelles.
J'ai atteint les objectifs que je m'étais fixés,
notamment en mettant en avant une observation originale dans les données et en
proposant des modèles et des implémentations de dernière génération pour
expliquer ces observations.

Mes objectifs futurs incluent la poursuite de mes études en neuroscience
computationnelle, notamment autour de la perception visuelle et de la rétine.
Je continue mes travaux en tant qu'ingénieur de recherche au sein de l'équipe
pour l'année à venir, avec pour projet de candidater à des bourses de thèses au
printemps prochain.

\clearpage

\end{document}