\section{Conclusion}\label{sec:Conclusion}

We combined large-scale recordings of RGC responses to natural movie
stimulations with CNN-based modeling to investigate the mechanisms of fast
contrast adaptation in the retina.

The modeling of retinal responses to natural stimuli has improved our
understanding of complex retinal processing. In a recent review, Karamanlis and
colleagues [ADD CITE], suggested three perspectives of study on the retinal
encoding of natural scenes: The circuit perspective ('How is the retinal code
implemented?'), the normative perspective ('Why is it complimented this way?)
and the coding perspective ('What is the code used by the retina?'). In this
work,
We focus on the 'what'. By exploring the response of the retina to a portion of
the spatio-temporal stimuli space we can gain insight into the code used by the
retina on that subspace. To explore further the 'how' perspective, one would
need to study how the different known types of cells in the retina participate
in that encoding. This poses the challenge of bridging the typing of cells from
functional and anatomical perspectives.
The normative perspective has also been explored using deep CNNs with
anatomically realistic constrained. It is likely that species with simpler
cortical circuitry, as mice, have a stronger need for upstream feature
extraction, in the retina. In opposition, species with computationally powerful
cortexes such as primates can deal with more faithful and linear
representations
of the visual inputs.
Some studies admirably developed approaches that allow investigation of retinal
processing from all three perspectives [ADD CITE].