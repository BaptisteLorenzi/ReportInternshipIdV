\section{Conclusion}\label{sec:Conclusion}

% Summary
% Adaptation depends on previous patterns
We observed that during natural image stimulation, many ganglion cells change
their selectivity for different parts of the images depending on previous light
patterns.

% Modeling and Simulation
We combined large-scale recordings of RGC responses to natural movie
stimulations with CNN-based modeling to investigate such mechanisms of fast
contrast adaptation in the retina.
% Well I don't really have the rest to say here since it's not done

% Interplay of different subunit pathways
Under realistic regularization constraints, the CNNs learned a structure
similar to retinal pathways, where a ganglion cell activation is the result of
a pooling of local subunits of different types in a specific area in space, the
ganglion cell receptive field.

% Neural cells encode spatio-temporal features of natural scenes
Our result supports the idea that retinal ganglion cells encode both
spatial and temporal features of natural scenes on a local scale.
Previous works have described those features to be encoded as features of the
retinal response (latency, firing rate). We might be able to support this
theory with further comparison of measured responses with predicted responses of
our model.

% Future Directions

TO DO


% Comparative analysis

% Utility of measurements tools
Maheswaranathan and colleagues \citep{maheswaranathan_interpreting_2023} have recently
been able to predict different aspect of encoding in the retina using deep
comvolutional network. In comparison, our experimental approach of estimating
the LSTA allow a direct comparison from the model to the data. Classical
estimation of performance can't describe what is missed in the prediction,
while our qualitative comparison might be able to.

% The what, why and how the retina
% I think it's and accessible illustration of the main questions regarding the retinal code
The modeling of retinal responses to natural stimuli has improved our
understanding of complex retinal processing. In a recent review, Karamanlis and
colleagues /citep{}, suggested three perspectives of study on the retinal
encoding of natural scenes: The circuit perspective ('How is the retinal code
implemented?'), the normative perspective ('Why is it complimented this way?)
and the coding perspective ('What is the code used by the retina?'). In this
work,
We focus on the 'what'. By exploring the response of the retina to a portion of
the spatio-temporal stimuli space we can gain insight into the code used by the
retina on that subspace. To explore further the 'how' perspective, one would
need to study how the different known types of cells in the retina participate
in that encoding. This poses the challenge of bridging the typing of cells from
functional and anatomical perspectives.
The normative perspective has also been explored using deep CNNs with
anatomically realistic constrained. It is likely that species with simpler
cortical circuitry, as mice, have a stronger need for upstream feature
extraction, in the retina. In opposition, species with computationally powerful
cortexes such as primates can deal with more faithful and linear
representations
of the visual inputs.
Some studies admirably developed approaches that allow investigation of retinal
processing from all three perspectives [ADD CITE].
