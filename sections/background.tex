\section{Background}\label{sec:background}

The retina is part of the central nervous system in vertebrates. It is
made of only a handful of layers of neurons. Its first layer is composed of
photo-sensitive neurons called photoreceptors, that act as light sensors for
the network. They give their excitatory output to bipolar cells, which can be
divided into 14 different types and each type responds differently to the same
stimulus, allowing for a vast functional diversity. Bipolar cells excite in
turn ganglion cells, which finally send the pre-processed visual information to
the rest of the brain through the optic nerve. Ganglion cells can also be
divided into different functional types (at least 32) and each type is believed
to extract a different feature from the visual scene. The retina also has two
classes of inhibitory neurons, horizontal and amacrine cells, that further
modulate the processing of excitatory cells. Compared to the rest of the brain,
its relative simplicity and its relatively easy experimental accessibility make
the retina an ideal neural tissue to study using computational models.

\textbf{Adaptation in the retina}
To operate optimally in a wide range of
stimulation conditions, the retina adapts its responses to the statistics of
the visual scene.
In particular, it was observed to adapt both to the average
luminance (stimulus average) and the average contrast (stimulus distance
from the average or variance).

Visual system can function over a wide range of light intensities, from
starlight to a bright sunny day – a luminance range of 10 10
The retinal adaptation to the luminance of the scene is quite simple by nature.
For instance, it is known that the retina uses different neuronal
pathways at low and high luminance. Rods and their retinal neuronal channels
cover the dimmest light while cones facilitate contrast, color and motion
discrimination but only in brighter light.

Contrast adaptation, by comparison, is harder to study. It was always studied
through the use of simple stimuli.
%clear, need proof
Contrast adaptation is known to have different timescales.
While slower contrast adaptation ($\approx10s$) is better understood, fast
adaptation (<1s) is more complex to study. It is still unclear how it
affects temporal processing and the sensitivity to stimulus
features\citep{baccus_fast_2002}. Furthermore, contrast adaptation can also
happen at different scales, either at the whole scene scale (global contrast
adaptation) or within one ganglion cell receptive field, the part of the visual
field that the cell receives inputs from (local contrast adaptation)
\citep{garvert_local_2013}. Local contrast adaptation is especially relevant in
understanding how ganglion cells respond to natural images since these stimuli
are full of spatial details like edges in which two contrast levels appear
simultaneously. Such images are challenging to use, as they can't be summed up
to a few statistics easily.