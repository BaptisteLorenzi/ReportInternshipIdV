\begin{abstract}
    Retina ganglion cells extract visual information from natural scenes.
    Not only can they extract spatial patterns but also temporal patterns
    thanks to diverse adaptation mechanisms.
    It is unclear how these mechanisms are active under natural scene
    simulation. Here we trained a convolutional neural network (CNN) model on
    large-scale
    functional recordings of RGC responses to natural mouse movies, and then
    used this model to investigate the role of past visual events in the
    activity of
    retinal ganglion cells.
    Our work showcases how a combination of experiments with natural stimuli
    and computational modeling allows the discovery of novel types of stimulus
    selectivity and
    build some hypotheses on how they are implemented.
\end{abstract}