% \appendix
\section{Enjeux technologiques, sociaux et éthiques}\label{app:enjeux}

Le travail que j'ai réalisé pendant ce stage est relativement fondamental.
Toutefois, la compréhension des mécanismes de la vision est un enjeu majeur à
la fois pour la recherche en neurosciences et dans une moindre mesure pour la
recherche en intelligence artificielle.

D'un point de vue médicale, le laboratoire dans lequel j'ai effectué mon stage
développe au moins deux projets visant les maladie de la vision.
Le premier en collaboration avec l'entreprise Essilor, vise à coomprendre
comment la rétine rentre en jeu dans le développement de la myopie. C'est un
enjeu médical
et industriel important car la rétine touche aujourd'hui plus de 20\% de la
population mondiale et ce chiffre est en constante augmentation.
Le second projet vise la resaturation partielle de la vision chez des patients
atteints de maladie dégénérative de la rétine (DMLA).
La plus ancienne des stratégies consiste en la pause d'implants rétiniens qui
viendraient copier l'activité du nerf optique.
Malgrè de nombreuse difficultés techniques, cette solution est en phase
d'essais cliniques dans des start up comme Pixium Vision
(https://www.pixium-vision.com/).
Une solution plus récente consiste à rendre sensible à la lumière (laser de
haute puissance) par optogénétique des
cellules ganglionnaires de la rétine.
Toutefois, ces cellules sont beaucoup moins nombreuses que les photorécepteurs
et la résolution de l'image est donc très faible.
Le projet du laboratoire vise à rendre sensible à la lumière des cellules
bipolaires qui sont plus nombreuses et qui sont connectées aux cellules
ganglionnaires.
Cela permettrait d'augmenter la résolution de l'image et donc de rendre la
vision retrouvée plus précise.
Les cellules bipolaires sont malheureusement beaucoup plus difficiles à cibler
car elles ne sont pas en surface de la rétine.
Une nouvelle technologie en cours de développement au laboratoire permettrait
égallement d'activer ces cellules non pas par lumière laser mais par des
vibrations sonores, beaucoup moins énergétiques.
Mes travaux participent entre autres à la compréhension du code neuronale de la
rétine. Un savoir nécessaire pour définir avec quel signal activer ces
cellules.

D'un point de vue technologique, le système cérébral est égallement un des
modéles les plus parfait de traduction de l'information visuelle.
Cette tâche nécessaire à de nombreuses applications en intelligence
artificielle (robotique, imagerie médicale, conduite autonome...), il est
égallement capable de la réaliser avec une efficacité énergétique et une
robustesse inégalée par la machine. La compréhension des mécanismes de la
vision est égallement un élément clef pour le développement de technologies
explicables et sans biais, un enjeu de poids en imagerie médicale notamment.

\section{Bilan de compétences}\label{app:bilan_competences}

Le présent bilan de compétences a pour objectif de récapituler les compétences
acquises et les expériences professionnelles vécues lors de mon stage.

\subsection{Compétences Techniques Acquises}

\begin{enumerate}
      \item \textbf{Analyse de Données:} J'ai acquis des compétences avancées
            en
            analyse de données de multi-eletrode array (MEA). J'ai appris à
            utiliser des outils tels que
            spiking circus, phy et la pipeline du laboratoire pour le
            prétraitement et l'analyse statistique des
            données. J'ai également contribué directement à l'amélioration de
            cette pipeline.

      \item \textbf{Modélisation Computationnelle:} J'ai développé des
            compétences en modélisation de réseaux neuronaux à l'aide de
            logiciels comme Pytorch.
            J'ai également
            travaillé sur l'implémentation de modèles mathématiques pour
            simuler
            des phénomènes neurologiques ayant un aspect à la fois spatial et
            temporel.
            J'ai également participer à la mise en place d'une pipeline
            d'optimisation d'entrainement de modèles
            (calcul parrallèle sur GPU, sauvegarde des modèles, sauvegarde des
            résultats...).

      \item \textbf{Programmation:} J'ai amélioré mes compétences en
            programmation et en visualisation des donnés, en particulier en
            Python et en utilisant des
            bibliothèques
            telles que NumPy, TensorFlow et PyTorch pour la mise en œuvre de
            modèles et d'algorithmes.

      \item \textbf{Analyse Statistique:} J'ai acquis une solide compréhension
            des méthodes statistiques utilisés pour évaluer les modèles de
            prédiction
            de réseaux de neurones biologiques (modèles de poisson, single-triral
            correlation...).

      \item \textbf{Communication Scientifique:} J'ai perfectionné mes
            compétences en rédaction de rapports scientifiques, en préparation
            de
            présentations, et en communication des résultats de mes recherches
            de
            manière claire et accessible. Durant mon stage j'ai participé au
            congrès junior
            de l'Université Paris Saclay. J'ai également pu présenter des
            travaux
            antérieurs au stage
            à l'Institut du Cerveau ainsi qu`à la Journée du département Science
            des Données du plateau de Saclay.

      \item \textbf{Culture scientifique:} Je suis resté dans une optique
            d'apprentissage constant en lisant
            de nombreux articles scientifiques et en participant à des
            séminaires
            et des
            conférences, notamment une semaine à l'"European Retina Meeting".
            J'ai également participé à des réunions de laboratoire
            hebdomadaires.

      \item \textbf{Suivi de projet de longue durée:} J'ai appris à gérer mon
            temps et à travailler de manière autonome sur un projet de longue
            durée, bien souvent en l'absence de deadline et d'objectifs
            concrets.
            Pour cela j'utilise dfférents outils de gestion de projet comme la
            tenue d'un journal, d'un agenda et d'un wiki personnel.

\end{enumerate}

\subsection{Expériences Clés}

\begin{enumerate}
      \item \textbf{Projet de Recherche:} J'ai participé activement à un
            projet
            de recherche sur la modélisation fonctionelle de la rétine, en
            contribuant à la conception
            expérimentale, à l'analyse des résultats et
            à la modélisation des observations.

      \item \textbf{Collaboration Interdisciplinaire:} J'ai eu l'occasion de
            collaborer avec des chercheurs issus de disciplines diverses comme
            la biologie, l'informatique et la physique, ce qui
            m'a permis
            d'élargir mes horizons et d'apporter une perspective complémentaire
            à mon travail.

      \item \textbf{Travail en Équipe Internationale:} J'ai travaillé en
            collaboration
            avec d'autres membres de l'équipe de recherche, ce qui m'a permis
            d'apprendre à
            résoudre des problèmes de manière collective et de développer des
            compétences interpersonnelles. J'ai appris à utiliser l'anglais au
            travail de manière quotidienne pour m'adapter à l'aspect
            international de l'équipe.
\end{enumerate}

\subsection{Conclusion}

En conclusion, mon stage de recherche de fin d'étude m'a
permis d'acquérir des compétences techniques avancées et une expérience
précieuse dans le domaine des neurosciences computationnelles.
J'ai atteint les objectifs que je m'étais fixés,
notamment en mettant en avant une observation originale dans les données et en
proposant des modèles et des implémentations de dernière génération pour
expliquer ces observations.

Mes objectifs futurs incluent la poursuite de mes études en neuroscience
computationnelle, notamment autour de la perception visuelle et de la rétine.
Je continue mes travaux en tant qu'ingénieur de recherche au sein de l'équipe
pour l'année à venir, avec pour projet de candidater à des bourses de thèses au
printemps prochain.