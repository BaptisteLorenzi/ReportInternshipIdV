\section{Introduction}
\label{sec:introduction}

The visual system processes complex stimuli by extracting from the visual scene spatio-temporal features that are relevant to behavior. This means that different neurons at different stages of perception are sensitive to specific features of the visual stimulus. Pulling together information from all this different types of neurons, the brain is then able to reconstruct the visual perception. This feature extraction already happens at the level of retina \citep{gollisch_eye_2010}. % However, it has been studied in the retina mostly in the case of simple stimuli like full-field flashes or moving bars. How the retina processes more complex stimuli like natural scenes is conversely still largely unclear.

Considered a part of the central nervous system in vertebrates, the retina is made of only a handful of layers of neurons. The first layer is composed of photo-sensitive neurons called photoreceptors, that act as light sensors for the network. They give their excitatory output to bipolar cells, which can be divided into 14 different types and each type responds differently to the same stimulus, allowing for a vast functional diversity. Bipolar cells excite in turn ganglion cells, which finally send the pre-processed visual information to the rest of the brain through the optic nerve. Ganglion cells can also be divided into different functional types (at least 32) and each type is believed to extract a different feature from the visual scene. The retina also has two classes of inhibitory neurons, horizontal and amacrine cells, that further modulate the processing of excitatory cells. Compared to the rest of the brain, its relative simplicity and its relatively easy experimental accessibility make the retina an ideal neural tissue to study using computational models. \newline

It has been shown that in order to operate optimally in a wide range of stimulation conditions, the retina adapts its responses to the statistics of the visual scene. In particular, it was observed to adapt both to the average luminance (stimulus average) and to the average contrast (stimulus distance from the mean or variance). 
%Comment: I think the question is not phrased very well. Contrast adaptation is also relatively well known as well. The question is how does fast contrast adaptation plays a role in natural scene/image processing. 
The retinal adaptation to the luminance of the scene is somehow well understood. For instance, it is known that the retina uses different neuronal pathways at low and high luminance. Less attention was given to the contrast adaptation, and it was always studied through the use of simple stimuli. From previous studies, contrast adaptation is known to have different timescales. While slower contrast adaptation ($\approx10s$) is better understood, fast adaptation (<1s) is more complex to study, and it is still unclear how it affects temporal processing and the sensitivity to stimulus features\citep{baccus_fast_2002}. Furthermore, contrast adaptation can also happen at different scales, either at the whole scene scale (global contrast adaptation) or within one ganglion cell receptive field, the part of the visual field that the cell receives inputs from (local contrast adaptation) \citep{garvert_local_2013}. Local contrast adaptation is especially relevant in understanding how ganglion cells respond to natural images, since these stimuli are full of spatial details like edges in which two contrast levels appear simultaneously. 
%SAM: careful, here you speak of contrast as distance from the grey while before you spoke of it as stimulus variance. They are similar but not exactly the same. It can be confusing
Such images are challenging to use, as they can't be summed up to a few statistics easily. 

In this work, we want to address how does fast contrast adaptation plays a role in natural scene processing. 

%SAM: I have changed the wording  and added something up to here but I believe the structure should be reorganized a bit. If you are not speaking about natural images up to here, why cite them in the first sentence. Also, you repeat several times in the introduction what you want to do changing it each time. State it only once at the end